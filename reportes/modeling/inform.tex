%% Estructura principal para un reporte de Trabajos intersemanales CIRCAE %%
\documentclass[a4paper]{IEEEtran} %tamaño del papel y el tipo de transcripción que será IEEE
\usepackage[utf8]{inputenc} %el tipo de codificación que incluye símbolos como la tilde
\usepackage[spanish]{babel} % hacemos que nuestro documentación vaya en español
\usepackage{cite} % citas bibliográficas
\usepackage{graphicx} %gráficos, usaremos solo .jpg o .png con estándares que ya veremos
\usepackage{subfigure}
\usepackage{url}
\usepackage{amsmath}
\usepackage{booktabs} 
\providecommand{\keywords}[1]{\textbf{\textit{Términos Clave---}} #1}
\begin{document}
%\tableofcontents%tabla de contenidos
%\listoffigures%lista de figuras
\title{Esquemas de Control Tipo Hiperbólico para Aeropéndulo}
\author{Hanan Ronaldo Quispe Condori, Joel Huamán Zárate, Héctor Lionel García Hurtado, Roly Sandro Gutierrez Benito, María Teresa Vera Percca}
%\markboth{INFORME CIRCAE 2019-08-05-G1-P3-001}{} % Codigo del informe que corresponde a: semestre | mes | dia | numero de grupo con la G antepuesta | numero de proyecto con la P antepuesta | número de informe
\maketitle
\begin{abstract}
    Se propone la aplicación de una familia de esquemas de control tipo hiperbólico de alto desempeño para el problema de control de posición angular del aeropendulo, los metodos de control clasicos resuelven este problema satisfactoriamente en el sistema linealizado mas el enfoque no lineal permite un control preciso sobre toda la dinámica del sistema en mención.
\end{abstract}
\section{Introducción}
kblalblkajfd\cite{gunnel2017tuning}
\section{Modelamiento Matemático}
\label{sec:modeling}
Se usará el enfoque de dinámica analítica, este enfoque usa las ecuaciones de movimiento de Euler-Lagrange para la obtención del modelo dinámico del sistema, se eligio este metodo debido a que se obtiene una estructura matemática bien definida el cual se integra de manera natural con la teoria de estabilidad de Lyapunov para el análisis de estabilidad para sistemas dinámicos lineales y no lineales sin importar el orden del sistema.[\cite{reyes2019drones}]

Se obtendrá la cinemática directa del sistema en función de la coordenada generalizada $\  theta$.

\begin{equation}
    \begin{bmatrix}  x \\ y \\
    \end{bmatrix}=
    \begin{bmatrix} Lsen(\theta) \\ -L\cos(\theta)  \\
    \end{bmatrix}
    \label{eq:cin_direc}
\end{equation}

Apartir de \ref{eq:cin_direc} se podrá obtener la cinemática diferencial para calcular la rapidez lineal.

\begin{equation}
    \frac{d}{dt}\begin{bmatrix}
        x \\
        y \\
    \end{bmatrix}=
    \begin{bmatrix}
        \dot{\theta}L\cos(\theta)  \\
        \dot{\theta}L\sen(\theta)  \\
    \end{bmatrix}
    \label{eq:speed}
\end{equation}

La rapidez lineal esta dada por

\begin{equation}
    \begin{split}
        \|v\|^2&=v.v^T=\sqrt{\dot{\theta}^2L^2sen^2{\theta}+\dot{\theta}^2L^2cos^2{\theta}}^2\\
        \|v\|^2&=L^2\dot{\theta}^2
    \end{split}
    \label{eq:speed_mod}
\end{equation}

Usando la expresión de la rapidez lineal del sistema se procedera a calcular el modelo de energía del sistema.

\begin{equation}
    \begin{split}
        \mathcal{K}(\theta,\dot{\theta}) &= \frac{1}{2}mL^2\dot\theta^2\\
        \mathcal{K}(\theta,\dot{\theta}) &= \frac{1}{2}J\dot\theta^2\\
        \mathcal{U}(\theta) &=  -m_1g\frac{L}{2}\cos{\theta} -m_2gL\cos{\theta} \\
        \mathcal{U}(\theta) &= -gL\cos{\theta(\frac{m_1}{2}+m_2)}
    \end{split}
    \label{eq:energy}
\end{equation}

El lagrangiano del sistema esta dado por
\begin{equation}
    \begin{split}
        \mathcal{L}(\theta,\dot{\theta})=\mathcal{K}(\theta,\dot{\theta})-\mathcal{U}(\theta)=\frac{1}{2}J\dot{\theta}^2+gL\cos{\theta}(\frac{m_1}{2}+m_2)
    \end{split}
    \label{eq:lagrange}
\end{equation}

Las ecuaciones de Euler lagrange para el sistema tienen la siguiente forma

\begin{equation}
    \frac{d}{dt}\begin{bmatrix}
        \frac{\partial \mathcal{L}(\theta,\dot{\theta})}{\partial\dot{\theta}}
    \end{bmatrix}-\begin{bmatrix}
        \frac{\partial \mathcal{L}(\theta,\dot{\theta})}{\partial \theta}
    \end{bmatrix}+f_f(f_e,\dot{\theta})=\tau
    \label{eq:eu_lagran}
\end{equation}

Usando el lagrangiano se tienen las siguientes expresiones

\begin{equation}
    \begin{split}
        \begin{bmatrix}
            \frac{\partial \mathcal{L}(\theta,\dot{\theta})}{\partial\dot{\theta}}
        \end{bmatrix}&=J\dot{\theta}\\
        \frac{d}{dt}\begin{bmatrix}
            \frac{\partial\mathcal{L}(\theta,\dot{\theta})}{\partial\dot{\theta}}
        \end{bmatrix}&=J\ddot{\theta}\\
        \begin{bmatrix}
            \frac{\partial \mathcal{L}(\theta,\dot{\theta})}{\partial \theta}
        \end{bmatrix}&=-Lg\sen(\theta)(m_2 + \frac{m_1}{2})\\
    \end{split}
    \label{eq:dq}
\end{equation}

El modelo dinámico del sistema considerando el fenomeno de fricción esta dado por:

\begin{equation}
    J\ddot{\theta}+Lg\sen(\theta)(m_2+\frac{m_1}{2})+c\dot{\theta}=\tau
    \label{eq:din_model}
\end{equation}

Además del modelo dinámico del pendulo tambien se tendrá que obtener un modelo dinámico del rotor, esto para obtener un modelo que describa en su totalidad el comportamiento físico del sistema.

Típicamente 

%\vspace{10mm}
\bibliographystyle{ieeetr}
\bibliography{bibliografia}
\end{document}